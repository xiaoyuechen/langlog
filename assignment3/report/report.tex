\documentclass[a4paper]{article}

\usepackage{amsmath, amssymb}
\usepackage{logicproof}

\author{Xiaoyue Chen}
\title{1DV517\\Assignment 3 report}

\begin{document}
\maketitle

\section{}
\subsection{}
Denote A is red as $a$, then A is blue is
$\neg a$, and A is not blue is $\neg \neg a$ which is
$a$. Similarly, denote B, C, D, E is red as
$b,c,d,e$ respectively. We have

\begin{itemize}
	\item $\lnot c \lor b$
	\item $(d \lor e) \land (\lnot d \lor \lnot e)$
	\item $a \rightarrow d$
	\item $e \equiv c$
	\item $b \rightarrow (a \land c)$
\end{itemize}

\subsection{}
\begin{logicproof}{2}
	\lnot c \lor b & premise \\
	(d \lor e) \land (\lnot d \lor \lnot e) & premise \\
	a \rightarrow d & premise \\
	e \equiv c & premise \\
	b \rightarrow (a \land c) & premise \\
	\begin{subproof}
		c & assumption \\
		b & disjunctive syllogism 1,6 \\
		a \land c & $\rightarrow$e 5,7 \\
		a & $\land$e 8 \\
		d & $\rightarrow$e 3,18 \\
		e & $\equiv$e 4,6 \\
		\lnot d \lor \lnot e & $\land$e 2 \\
		\lnot e & disjunctive syllogism 10,12 \\
		\bot & $\lnot$e 11,13
	\end{subproof}
	\lnot c & $\lnot$i 6--14 \\
	\lnot e & $\equiv$e 4,15 \\
	d \lor e & $\land$e 2 \\
	d & disjunctive syllogism 16,17 \\
	\begin{subproof}
		a \land c & assumption \\
		c & $\land$e 19 \\
		\bot & $\lnot$e 15,20
	\end{subproof}
	\lnot (a \land c) & $\lnot$i 19--21 \\
	\lnot b & MT 5,22 \\
	(\lnot b \land \lnot c \land d \land \lnot e) & $\land$i
	15,16,18,23
\end{logicproof}
Hence the possible colors of each box are
\begin{itemize}
	\item A: red or blue
	\item B: blue
	\item C: blue
	\item D: red
	\item E: blue
\end{itemize}

\section{}
\subsection{}
Predicates
\begin{itemize}
	\item $P(x)$: $x$ is a program
	\item $M(x,y)$: $y$ is a method of
	      $x$
	\item $I(x,y)$: $x$ can invoke
	      $y$
	\item $E(x,y)$: $y$ is an entry point of
	      $x$
	\item $U(x)$: $x$ is a user
	\item $S(x,y)$: $y$ is a sink of
	      $x$
	\item $A(x,y)$: $y$ is accessible from
	      $x$
\end{itemize}

\subsection{}
\begin{itemize}
	\item
	      We define a program as a collection of methods invoking each other.
	      \begin{equation}
		      \label{eq:program}
		      \forall p.(P(p) \equiv \exists m_1,m_2.(M(x,m_1) \land M(x,m_2)
		      \land I(m_1, m_2)))
	      \end{equation}
	\item
	      An entry point is a method called by the user to run the program. An entry
	      point cannot be invoked by other methods.
	      \begin{equation}
		      \label{eq:entry}
		      \forall e.(E(e) \rightarrow (M(e) \land \exists u.(U(u) \land
		      I(u,e)) \land \forall m.(m \neq e \rightarrow \lnot I(m,e))))
	      \end{equation}
	\item
	      Each program has exactly one entry point.
	      \begin{equation}
		      \forall p.(P(p) \rightarrow \exists e.(H(p,e) \land E(e)
		      \land \forall e'.(H(p,e') \land E(e') \rightarrow e=e'))
	      \end{equation}
	\item
	      A program should have some sinks that are accessible from the entry point.
	      \begin{equation}
		      \forall p.(P(p) \rightarrow \exists s.(S(s) \land H(p,s)
		      \land \exists e.(E(e) \land H(p,e) \land A(e,s))))
	      \end{equation}
\end{itemize}

\end{document}

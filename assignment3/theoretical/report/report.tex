\documentclass[a4paper]{article}

\usepackage{amsmath, amssymb}
\usepackage{logicproof}

\author{Xiaoyue Chen}
\title{1DV517\\Assignment 3 report}

\begin{document}
\maketitle

\section{}
\subsection{}
Denote A is red as $a$, then A is blue is
$\neg a$, and A is not blue is $\neg \neg a$ which is
$a$. Similarly, denote B, C, D, E is red as
$b,c,d,e$ respectively. We have

\begin{itemize}
	\item $\lnot c \lor b$
	\item $(d \lor e) \land (\lnot d \lor \lnot e)$
	\item $a \rightarrow d$
	\item $e \equiv c$
	\item $b \rightarrow (a \land c)$
\end{itemize}

\subsection{}
\begin{logicproof}{2}
	\lnot c \lor b & premise \\
	(d \lor e) \land (\lnot d \lor \lnot e) & premise \\
	a \rightarrow d & premise \\
	e \equiv c & premise \\
	b \rightarrow (a \land c) & premise \\
	\begin{subproof}
		c & assumption \\
		b & disjunctive syllogism 1,6 \\
		a \land c & MP 5,7 \\
		a & $\land$e 8 \\
		d & MP 3,18 \\
		e & equivalent 4,6 \\
		\lnot d \lor \lnot e & $\land$e 2 \\
		\lnot e & disjunctive syllogism 10,12 \\
		\bot & $\lnot$e 11,13
	\end{subproof}
	\lnot c & $\lnot$i 6--14 \\
	\lnot e & equivalent 4,15 \\
	d \lor e & $\land$e 2 \\
	d & disjunctive syllogism 16,17 \\
	\begin{subproof}
		a \land c & assumption \\
		c & $\land$e 19 \\
		\bot & $\lnot$e 15,20
	\end{subproof}
	\lnot (a \land c) & $\lnot$i 19--21 \\
	\lnot b & MT 5,22 \\
	(\lnot b \land \lnot c \land d \land \lnot e) & $\land$i
	15,16,18,23
\end{logicproof}
Hence the possible colors of each box are
\begin{itemize}
	\item A: red or blue
	\item B: blue
	\item C: blue
	\item D: red
	\item E: blue
\end{itemize}

\section{}
\subsection{}
Predicates
\begin{itemize}
	\item $P(x)$: $x$ is a program
	\item $M(x)$: $x$ is a method of
	\item $I(x,y)$: $x$ can invoke
	      $y$
	\item $E(x)$: $x$ is an entry point
	\item $U(x)$: $x$ is a user
	\item $S(x)$: $x$ is a sink
	\item $A(x,y)$: $y$ is accessible from
	      $x$
	\item $NV(x)$: $x$ is an invalid program
\end{itemize}
Formula
\begin{itemize}
	\item
	      We define a program as a collection of methods invoking each other.
	      \begin{equation}
		      \label{lgc:program}
		      P(p) \equiv \exists m_1,m_2.(M(m_1) \land M(m_2) \land I(m_1, m_2))
	      \end{equation}
	\item
	      An entry point is a method called by the user to run the program. An entry
	      point cannot be invoked by other methods.
	      \begin{equation}
		      \label{lgc:entry_point}
		      \forall e.(E(e) \rightarrow (M(e) \land \exists u.(U(u) \land
		      I(u,e)) \land \forall m.(m \neq e \rightarrow \lnot I(m,e))))
	      \end{equation}
	\item
	      Each program has exactly one entry point.
	      \begin{equation}
		      \label{lgc:one_entry_point}
		      P(p) \rightarrow \exists e.(E(e) \land \forall e'.(E(e') \rightarrow e=e')
	      \end{equation}
	\item
	      Define accessible
	      \begin{equation}
		      \label{lgc:accessible}
		      \forall x,y.(A(x,y) \equiv \exists z_1, z_2, \cdots, z_n.(I(x,z_1) \land
		      I(z_1, z_2) \land \cdots \land I(z_n, y)))
	      \end{equation}
	\item
	      A program should have some sinks that are accessible from the entry point.
	      \begin{equation}
		      \label{lgc:sink}
		      P(p) \rightarrow \exists s.(S(s) \land \exists e.(E(e) \land A(e,s))))
	      \end{equation}
	\item
	      A program is invalid, if it does not satisfy any of the above conditions.
	      \begin{equation}
		      \label{lgc:valid}
		      \forall p.(\lnot (\eqref{lgc:program} \land \eqref{lgc:entry_point} \land
		      \eqref{lgc:one_entry_point} \land \eqref{lgc:accessible} \land
		      \eqref{lgc:sink}) \rightarrow NV(p))
	      \end{equation}
\end{itemize}

\subsection{}
Predicate
\begin{itemize}
	\item $W(x)$: $x$ is an unused method
\end{itemize}
Formula
\begin{equation}
	\forall w.(W(w) \equiv (M(w) \land
	\forall e.(E(e) \rightarrow \lnot A(e, w) \land w \neq e))
\end{equation}

\subsection{}
To prove that any accessible method from \textbf{only} an unused
method is unused, we need to prove
\begin{multline}
	\forall m.((M(m) \land \exists w.(W(w) \land I(w,m) \\
	\land \forall m'.((I(m', m) \rightarrow m'=w)))
	\rightarrow W(m))
\end{multline}
To simplify the proof, let $m_0$ denote a method that is
accessible from \textbf{only} an unused method
$w_0$. Let $e_0$ denote the program entry
point. We need to prove
\begin{equation}
	I(w_0, m_0) \land \forall m.(I(m, m_0) \rightarrow m = w_0) \rightarrow W(m_0)
\end{equation}
Suppose $m_0$ is not unused, then $m_0$
could be accessed from $e_0$. $m_0$ could
only be accessed from $w_0$, hence only
$m_0$ could only be invoked from $w_0$.
$w_0$ is unused hence not accessible from
$e_0$ and is not $e_0$. We now use
contradiction to proof $m_0$ cannot be accessed from
$e_0$:
\begin{logicproof}{1}
	M(m_0) & premise \\
	\lnot A(e_0, w_0) & premise \\
	E(e_0) & premise \\
	I(w_0, m_0) & premise \\
	\forall m.(I(m, m_0) \rightarrow m = w_0) & premise \\
	\forall x,y.(A(x,y) \equiv \exists z_1, z_2, \cdots, z_n.(I(x,z_1) \land
	I(z_1,
	z_2) \land \cdots \land I(z_n, y)))
	& premise \\
	\begin{subproof}
		A(e_0, m_0) & assumption \\
		A(e_0, m_0) \equiv
		\exists z_1, z_2, \cdots, z_n.(I(e_0,z_1) \land I(z_1,
		z_2) \land \cdots \land I(z_n, m_0))
		& $\forall$e 6 \\
		\exists z_1, z_2, \cdots, z_n.(I(e_0,z_1) \land I(z_1,
		z_2) \land \cdots \land I(z_n, m_0))
		& equivalent 7,8 \\
		I(e_0,z_1) \land I(z_1, z_2) \land \cdots \land I(z_n, m_0)
		& $\exists$e 9 \\
		I(z_n, m_0) & $\land$e 10 \\
		I(z_n, m_0) \rightarrow z_n = w_0 & $\forall$e 5 \\
		z_n = w_0 & MP 11,12 \\
		I(e_0,z_1) \land I(z_1, z_2) \land \cdots \land
		I(z_{n-1}, z_n)
		& $\land$e 10 \\
		I(e_0,z_1) \land I(z_1, z_2) \land \cdots \land
		I(z_{n-1}, w_0)
		& equivalent 13,14 \\
		\exists z_1, \cdots, z_{n-1}.(I(e_0,z_1) \land I(z_1,
		z_2) \land \cdots \land
		I(z_{n-1}, w_0))
		& $\exists$i 15 \\
		A(e_0, w_0) \equiv
		\exists z_1, \cdots, z_{n-1}.(I(e_0,z_1) \land I(z_1,
		z_2) \land \cdots \land I(z_{n-1}, w_0))
		& $\forall$e 6 \\
		A(e_0, w_0) & equivalent 16,17 \\
		\bot & $\lnot$e 2,18
	\end{subproof}
	\lnot A(e_0, m_0) & PBC 7--19
\end{logicproof}
Hence any accessible method from \textbf{only} an unused method is
unused.

\subsection{}
We need to prove
\begin{equation}
	\forall s,p.((S(s) \rightarrow W(s)) \rightarrow NV(p))
\end{equation}
Suppose the program entry point is $e_0$
\begin{logicproof} {2}
	\forall s.(S(s) \rightarrow W(s)) & premise \\
	\forall w.(W(w) \equiv \lnot A(e_0, w) \land w \neq e_0) & premise \\
	\lnot \exists s.(S(s) \land A(e_0, s)) \rightarrow NV(p) & premise \\
	\begin{subproof}
		s_0 & \\
		\begin{subproof}
			S(s_0) \land A(e_0, s_0) & assumption \\
			S(s_0) \rightarrow W(s_0) & $\forall$e 1 \\
			S(s_0) & $\land$e 5 \\
			W(s_0) & MP 6,7 \\
			W(s_0) \equiv \lnot A(e_0, s_0) \land s_0 \neq e_0
			& $\forall$e 3 \\
			\lnot A(e_0, s_0) \land s_0 \neq e_0 & equivalent 6,7 \\
			\lnot A(e_0, s_0) & $\land$e 10 \\
			A(e_0, s_0) & $\land$e 5 \\
			\bot & $\lnot$e 11,12
		\end{subproof}
		\lnot (S(s_0) \land A(e_0, s_0)) & PBC 5--13
	\end{subproof}
	\forall s.\lnot (S(s) \land A(e_0, s)) & $\forall$i 4--14 \\
	\lnot \exists s.(S(s) \land A(e_0, s)) & $\forall \lnot \equiv \lnot \exists$ 15 \\
	NV(p) & MP 3, 16 \\
	\forall s.(S(s) \rightarrow W(s)) \rightarrow NV(p) & $\rightarrow$i
	1,17 \\
	\forall s,p.((S(s) \rightarrow W(s)) \rightarrow NV(p)) &
	$\forall$i 18
\end{logicproof}

\section{}
\subsection{}
\begin{logicproof}{2}
	\forall x.P(x) \rightarrow S & premise \\
	\begin{subproof}
		\forall x.P(x) & assumption \\
		S & MP 1,2 \\
		\begin{subproof}
			x_0 & \\
			P(x_0) & $\forall$e 2 \\
			P(x_0) \rightarrow S & $\rightarrow$i 3,5
		\end{subproof}
		\exists x.(P(x) \rightarrow S) & $\exists x$ i 4--6
	\end{subproof}
	\begin{subproof}
		\lnot (\forall x.P(x)) & assumption \\
		\exists x.\lnot P(x) & $\lnot \forall \equiv \exists \lnot$ 8 \\
		\begin{subproof}
			x_0 & \\
			\lnot P(x_0) & assumption \\
			\lnot P(x_0) \lor S & $\lor$i 11 \\
			P(x_0) \rightarrow S & equivalent 12,13 \\
			\exists x.(P(x) \rightarrow S) & $\exists x$ i 10--13
		\end{subproof}
		\exists x.(P(x) \rightarrow S) & $\exists x$ e 9, 10--14
	\end{subproof}
	\exists x.(P(x) \rightarrow S) & PBC 2--15
\end{logicproof}

\subsection{}
\begin{logicproof}{2}
	P(b) & premise \\
	\forall x \forall y.(P(x) \land P(y) \rightarrow x = y) & premise \\
	\forall x.(P(x) \land P(b) \rightarrow x = b) & $\forall$e 2 \\
	\forall x.(P(x) \rightarrow x = b) & $\land$e 1,3 \\
	\begin{subproof}
		x_0 & \\
		\begin{subproof}
			x_0 = b & assumption \\
			P(x_0) & equivalent 1,6
		\end{subproof}
		x_0=b \rightarrow P(x_0) & $\rightarrow$i 6,7 \\
		P(x_0) \rightarrow x_0 = b & $\forall$e 4 \\
		P(x_0) \leftrightarrow x_0 = b & $\land$i 8,9
	\end{subproof}
	\forall x.(P(x) \leftrightarrow x = b) & $\forall$i 5--10
\end{logicproof}

\subsection{}
\begin{logicproof}{2}
	\exists x,y.(H(x,y) \lor H(y,x)) & premise \\
	\lnot \exists x.H(x,x) & premise \\
	\forall x.\lnot H(x,x) & equivalent 2 \\
	\begin{subproof}
		\lnot \exists x,y.\lnot(x=y) & assumption \\
		\forall x,y.\lnot \lnot(x=y) & equivalent 4 \\
		\forall x,y.(x=y) & $\lnot \lnot$e 5 \\
		\begin{subproof}
			x_0, y_0 & \\
			H(x_0, y_0) \lor H(y_0, x_0) & assumption \\
			x_0 = y_0 & $\forall$e 6 \\
			\lnot H(x_0, x_0) & $\forall$e 3 \\
			\lnot H(x_0, y_0) & equivalent 9,10 \\
			\lnot H(y_0, x_0) & equivalent 9,10 \\
			\lnot H(x_0, y_0) \land \lnot H(y_0, x_0) & $\land$i 11,12 \\
			\lnot (H(x_0, y_0) \lor H(y_0, x_0)) & DeMorgan 13 \\
			\bot & $\lnot$e 8,14
		\end{subproof}
		\bot & $\exists$e 1, 7--15
	\end{subproof}
	\exists x,y.\lnot(x=y) & PBC 4--16
\end{logicproof}

\subsection{}
\begin{logicproof}{3}
	\forall x.P(a,x,x) & premise \\
	\forall x,y,z.(P(x,y,z) \rightarrow P(f(x),y,f(z))) & premise \\
	P(a, f(a), f(a)) & $\forall$e 1 \\
	P(a, f(a), f(a)) \rightarrow P(f(a), f(a), f(f(a)))
	& $\forall$e 2 \\
	P(f(a), f(a), f(f(a))) & MP 3.4 \\
	\begin{subproof}
		z_0 & \\
		z_0 = f(a) & assumption \\
		P(f(a), z_0, f(f(a)) & equivalent 5, 7 \\
		\exists z.P(f(a), z, f(f(a))) & $\exists$i 6--8
	\end{subproof}
	\exists z.(z=f(a)) & premise \\
	\exists z.P(f(a), z, f(f(a))) & $\exists$e 10, 6--9
\end{logicproof}

\end{document}
